\section{Opis założeń projektu}
\subsection{Cel projektu}
Głównym celem projektu było stworzenie wirtualnego automatu z napojami, który:
\begin{itemize}
\item Symuluje rzeczywiste zachowanie automatu vendingowego
\item Umożliwia łatwe zarządzanie produktami przez administratora
\item Zapewnia intuicyjny interfejs dla użytkowników
\item Rejestruje historię transakcji
\item Generuje raporty finansowe
\end{itemize}




\subsection{Wymagania funkcjonalne}
\begin{itemize}
\item Przeglądanie dostępnych produktów z podziałem na kategorie
\item System koszyka zakupowego
\item Obsługa płatności gotówką i kartą
\item Panel administracyjny z autentykacją
\item Zarządzanie stanem magazynowym
\item Generowanie raportów transakcji
\item Kalkulacja zysków 
\end{itemize}


\subsection{Wymagania niefunkcjonalne}
\begin{itemize}
\item Wydajność: czas reakcji < 1s dla podstawowych operacji
\item Bezpieczeństwo: ochrona danych transakcji
\item Kompatybilność: Java 8+, Windows/Linux/MacOS
\item Użyteczność: intuicyjny interfejs graficzny
\item Niezawodność: odporność na błędy użytkownika
\end{itemize}